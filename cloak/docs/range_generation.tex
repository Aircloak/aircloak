\documentclass[english,a4paper]{article}
\usepackage{babel,a4wide,dsfont}
\usepackage[utf8]{inputenc}
\parskip=\medskipamount
\parindent=0pt
\begin{document}
  \title{Range generation algorithm}
  \author{Aircloak}
  \maketitle

  This document gives an overview of the algorithm used to generate the reported ranges for a given property
  label/string pair.
  The basic idea is a sweep-line algorithm where we sweep over the different value points.
  
  \begin{description}
    \item[Input:] List of pair $((v_1,c_1),(v_2,c_2),\ldots,(v_n,c_n))$ where $v_i$ represents the value and
    $c_i$ the amount of users that reported this value.  The sequence has to be sorted, such that $\forall
    i=1,2,\ldots,n-1:v_i<v_{i+1}$.

    \item[Output:] List of ranges where each range is given in the form $(\min_i,\max_i,c_i,c_i')$ which
    represents the interval $[\min_i,\max_i)$ containing $c_i$ users.  $c_i'$ is the noisy count in the
    interval, thus applying $\delta_{T_1}$ (see the TÜV document describing the anonymization function).

    \item[Steps:] \mbox{}
    \begin{itemize}
      \item Find the minimum and maximum values $\min,\max$ of all values $v_1,v_2,\ldots,v_n$.  These are
      $\min=v_1$ and $\max=v_n$.

      \item Find the smallest $k\in\mathds{N}$ such that $[\min,\max]\subseteq[-2^k,2^k)$.

      \item Generate the initial set of working ranges.  This is stored as a list of the form
        $$((2^0, -2^k, -2^k+2^0, 0), (2^1, -2^k, -2^k+2^1, 0), \ldots, (2^(k+1), -2^k, 2^k, 0)).$$
        Each element is represented by its size, the range's lower and upper bounds, and the currently known
        number of users belonging to that range.

      \item Loop through the values from $1$ to $n$ ($i=1,2,\ldots,n$).  Each value is an event for the
      sweep-line algorithm.  For each event go through the list of working ranges beginning with the smallest
      range.  We have the following kind of events for each range $(s_r, \min_r, \max_r, C_r)$:
      \begin{enumerate}
        \item \label{event-type-1}$v_i<\min_r$:  do nothing.  The value is in front of the range.
        \item \label{event-type-2}$\min_r\le v_i<\max_r$:  update $C_r\leftarrow C_r + c_i$ (the value is in
          the range, so update their values).
        \item \label{event-type-3}$\max_r\le v_i$:  no new value/count pair may be found that is in that
          range.  We can check if we want to report that range.  Test if $\delta_{T_1}(C_r)>T_1$.
          \begin{itemize}
            \item \emph{yes}:  report $(\min_r,\max_r,C_r,\delta_{T_1}(C_r))$ and move this range to the next
              position like in the ``\emph{no}''-case.  Update all working ranges including the range
              $[\min_r,\max_r)$ (before the position update) to a new position as in the ``\emph{no}''-case.
              These working ranges have to be moved at least by one position because they include the reported
              range.
            \item \emph{no}:  move the range by $l\cdot s_r$ ($l\in\mathds{N}$) positions to the right, such
              that $v_i$ is in that range: $\min_r\leftarrow\min_r+l\cdot s_r$, $\max_r\leftarrow\max_r+l$,
              $\min_r\le v_i<\max_r$
          \end{itemize}
      \end{enumerate}
    \end{itemize}
  \end{description}

  The order of the working ranges is important.  We need to process the small ranges first as we want to
  report the smallest possible ranges with the corresponding property ($\delta_{T_1}(C_r)>T_1$).
  
  \paragraph*{Optimization:}
  If a working range's start is greater than the maximum value $v_n$, we can remove that working range from
  the list as we will never have an event of type \ref{event-type-2} or \ref{event-type-3}.

  \paragraph*{Conjecture:}
  If a range $r$ of size $s_r$ is moved such that $v_n<\min_r$ (past the maximum value), then for all ranges
  $r'$ with $s_{r'}$
  $$
    v_n<\min_{r'}.
  $$
  This would allow a further optimization.
\end{document}
